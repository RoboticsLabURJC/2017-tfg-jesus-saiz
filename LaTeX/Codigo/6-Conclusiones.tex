\chapter{Conclusiones}\label{cap.conclusiones}
\hspace{1cm} Una vez expuesto el desarrollo fundamental de este trabajo, en este capítulo vamos a analizar si los objetivos planteados anteriormente se han cumplido. A su vez comentaremos las posibles líneas de desarrollo que partan de este trabajo. Como visión global podemos decir que el trabajo ha sido satisfactorio, ya que se ha conseguido un algoritmo que permita al dron despegar, realizar una ruta y aterrizar, todo ello de manera completamente autónoma, como se deseaba en un primer momento. 

\section{Conclusiones}
\hspace{1cm} En cuanto al análisis de los objetivos vamos a extraer conclusiones sobre cada uno de ellos superado:

\begin{enumerate}
	\item{\textbf{Adaptación e integración del componente Slam-Visualmarkers:} hemos ajustado el componente y lo hemos introducido por pasos para finalmente conseguir su correcto funcionamiento. La nueva versión,  la cual hemos usado en este TFG, está implementada en ROS y fruto de este TFG se ha añadido al componente oficial la interfaz del numero de balizas observadas, que permite medir indirectamente la incertidumbre de la estimación de posición.}	
	\item{\textbf{Recodificación e integración del despegue y aterrizaje visual:} El algoritmo permite que el dron despegue de forma controlada estabilizandose sobre la baliza arlequinada, evitando así que se desvíe por factores externos y asegurando que se inicia la ruta deseada con el dron completamente nivelado. También mencionar que el aterrizaje se ha simplificado para eliminar los temporizadores y se ha introducido en la nueva máquina virtual de estados.}
	\item{\textbf{Desarrollo de dos módulos de navegación basados en la posición absoluta del dron:} Hemos logrado desarrollar y comprobar que, tanto el método de pilotaje por puntos de paso, en el cual no se controla el giro, como el pilotaje por trayectorias continuas que se
adapta mejor trayectorias complejas y permite un control más fino de la navegación autónoma, funcionan correctamente, con errores mínimos y con una robustez elevada en cualquier tipo de situaciones, demostrando así que se ha creado una mejora con respecto a los anteriores sistemas de pilotaje creados.}
	\item{\textbf{Utilización de la herramienta Visual States:} Se ha logrado desarrollar completamente el algoritmo dentro de esta herramienta. Se ha creado el autómata necesario para pasar de un estado a otro mediante transiciones controladas e integrar con ello la navegación autónoma, el despegue y el aterrizaje.}	
	\item{\textbf{Validación experimental en entorno simulado Gazebo:} Todos los resultados obtenidos se han extraído a partir de las pruebas realizadas dentro de los mundos creados en este entorno de simulación tridimensional, tanto con experimentos con pruebas unitarias, como con pruebas integrales. Creando así por tanto ejemplos visuales de la robustez del algoritmo completo, con la utilización de todos los módulos para dotar al sistema de autonomía completa. Mejorando el pilotaje, ligeramente en precisión espacial pero notablemente en la realización de rutas y en el tiempo empleado para ello. Con todo se ha conseguido validar el funcionamiento integral del sistema desarrollado en simulación como un sistema estable y aunque el ruido de las medidas y el comportamiento de los soportes físicos pueden afectar el comportamiento del sistema, este ha probado ser lo suficientemente robusto para satisfacer las metas propuestas dentro de entornos reales.}
\end{enumerate}

\hspace{1cm} Al cumplir el objetivo principal de la aplicación y todos los subobjetivos podemos decir que se ha conseguido mejorar lo existente anteriormente en JdeRobot sobre navegación autónoma de drones. Ahora cuenta con un algoritmo que permite a un dron navegar de forma completamente autónoma desde el despegue hasta el aterrizaje, siguiendo una ruta previamente establecida y utilizando una autolocalización propia, únicamente colocando una serie de balizas y pulsando el botón de inicio.

\hspace{1cm} Antes de extraer las conclusiones sobre los objetivos en el capítulo \ref{cap.objetivos}, mencionar que para conseguirlos se ha hecho uso de un gran número de herramientas y tecnologías que han tenido que ser comprendidas y adaptadas para el propósito buscado. Todas las pruebas y los avances que se han ido desarrollando y validando se pueden ver en la wiki oficial del proyecto \footnote{\url{http://jderobot.org/Jsaizc-tfg}} y el código está accesible en github \footnote{\url{https://github.com/RoboticsURJC-students/2017-tfg-jesus-saiz}} 

\section{Trabajos futuros}
\hspace{1cm} El mundo de la robótica, de la visión artificial y dentro de ambos de los drones, está en constante desarrollo por lo que cada día surgen nuevas necesidades y aplicaciones posibles para este tipo de robots. Este proyecto ha supuesto un aporte, un paso adelante dentro de JdeRobot pero abre a su vez nuevas posibilidades en esta misma línea. A continuación, se detallan algunas de estas líneas de mejora, que han surgido al investigar las herramientas, las nuevas aplicaciones y al desarrollar nuestro propio algoritmo.

\begin{itemize}
	\item Lo primero sería probar el algoritmo en situaciones reales. Aunque hemos comprobado la robustez de algoritmo en entornos simulados y hemos incluido algunas funcionalidades para enfrentarse a entornos reales, no hemos podido comprobar situaciones como: la deriva propia del vehículo, que añade un mayor grado de complejidad al sistema de control; los sistemas de estabilización establecidos por el fabricante, que en ocasiones entran en conflicto con el control proporcionado por el componente; y el ruido de los sensores.
	\item Otra posible linea de investigación es el cambio del sistema de autolocalización visual basado en marcadores por una autolocalización basada en un sistema de odometría visual, evitando así el uso de balizas y localizándonos gracias a la percepción de objetos a nuestro alrededor pudiendo crear incluso mapas en 3D de lo que se encuentra a nuestro alrededor.
	\item Por último, sería interesante extender nuestro algoritmo para adaptarlo a otras necesidades reales como, por ejemplo, capturas de fotografías de puntos estratégicos. Esto puede ser de gran utilidad para su representación posterior en 3D y creación de mapas u objetos e incluso para tareas de vigilancia o monitorización de espacios.
\end{itemize}