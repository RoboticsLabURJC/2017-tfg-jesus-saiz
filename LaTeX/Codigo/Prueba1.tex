\documentclass{book}
% Dimensiones y márgenes----------------------------------------------
\usepackage[total={18cm,21cm},top=0cm, left=2cm]{geometry}
\parindent=0mm
% Otros paquetes -----------------------------------------------------
\usepackage{mathpazo} %fuente palatino
\usepackage{graphicx}
\usepackage{xcolor}
\usepackage{pstricks}
\usepackage[T1]{fontenc}
\usepackage[utf8]{inputenc} %
\usepackage[spanish]{babel} % Idioma español
\usepackage{latexsym,amsmath,amssymb,amsfonts,cancel}
\usepackage[shortlabels]{enumitem}
%---------------------------------------------------------------------
\begin{document}
\begin{titlepage}
	\begin{center}
		\vspace*{20mm}
		\begin{center}
			\includegraphics[width=0.4\linewidth]{img/logo.jpg}
		\end{center}
		\vspace{6.5mm}
		
		\fontsize{15.5}{14}\selectfont ESCUELA TÉCNICA SUPERIOR DE INGENIERÍA DE TELECOMUNICACIÓN
		\vspace{8mm}
		
		\fontsize{14}{14}\selectfont GRADO EN INGENIERÍA AEROESPACIAL EN AERONAVEGACIÓN
		
		\vspace{60pt}
		
		\fontfamily{lmss}\fontsize{15.7}{14}\selectfont \textbf{TRABAJO FIN DE GRADO} 
		
		\vspace{20mm}
		\begin{huge}
			DRONES 
		\end{huge}
		
		\vspace{40mm}
		
		\begin{large}
			Autor: Jesús Saiz Colomina
			
			Tutor: José María Cañas Plaza
			
			\vspace{7mm}
		\end{large}
		\begin{normalsize}
			Curso académico 2017/2018		
		\end{normalsize}
		\vspace{7mm}
	\end{center}
\end{titlepage}

\tableofcontents

\chapter{Introducción}
\section{Aplicaciones actuales}
Áerea y continuarioa escribiendo texto para este apartado
hasta que salte de linea todo justificado que no se como 
se pone todavia xd aunque creo que viene por defecto aaaa 
aaa ddd ddsjkfn sdkjf sdjfnkjasdf askdjg kjsdag aksjdgb
sdagnsdkjagbasd osdjgn
\section{Próximas aplicaciones}
\section{Robótica áerea URJC}
\section{...}

\chapter{Objetivos}
\section{Objetivo Inicial}
\section{Objetivo Final}
\section{Requisitos}
\section{Metodología}

\chapter{Plan de trabajo}
\section{Infraestructura}
\section{Jderobot (todo lo utilizado)}
\section{Código en python}
\section{Bibliotecas}
\section{Librerias}

\chapter{Desarrollo}
\section{JDeRobotAcademy}
\section{Diseño (mundos y visual)}
\section{Primer problema abordado (drone simple)}
\section{Creación de rutas (posicion dada)}
\section{Aterrizaje visual}
\section{Anexionar lo anterior en un solo programa}
\section{Crearlo con VisualStates}
\section{Cambiar localicalización dada por slam-visualmarkers}
\section{Dividirlo en autolocalización y control ?}

\chapter{Experimentos (de momento todos simulados)}

\chapter{Conclusiones y trabajos futuros (según hasta donde llegue)}

\chapter{Bibliografía ?}

\chapter{Anexos ?}

\end{document}
